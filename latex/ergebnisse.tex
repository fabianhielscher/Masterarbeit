\chapter{Ergebnisse}
In diesem Kapitel werden die Ergebnisse der in Kapitel 2 beschriebenen Versuche präsentiert. Die Versuche wurden bei Raumtemperaturen von ca. $22^{\circ}$ Celsius und gewöhnlichen Lichtbedingungen durchgeführt. Die Werte für $r_{max}$, $v_{max}$ und $k$ sind für alle Steuerungsvarianten in allen Versuchen wie folgt definiert:
\begin{itemize}
	\item
	$r_{max} = 400 mm$
	\item
	$v_{max} = 0,25 \frac{mm}{ms}$
	\item
	$k = 0,2$
\end{itemize}

In einigen Ergebnissen wird für die statistische Auswertung der RMSE genutzt, welcher die Wurzel der gemittelten Fehlerquadratsumme darstellt und in Formel \ref{eq:rmse} definiert ist. Der Fehler einer Messung $err_j$ beschreibt die Differenz zwischen Soll- und Istwert. Um den Lesefluss nicht zu behindern werden die Bezeichnungen der Steuerungen bei einigen Versuchen abgekürzt.
\begin{equation}\label{eq:rmse}
RMSE=\sqrt{\frac{\sum_{j=1}^{n}\left(err_j\right)^2}{n}}
\end{equation}

\section{Tremor}

\begin{figure}[htb]
	\centering
	\includegraphics[width=0.9\textwidth]{plot_tremor.pdf}
	\caption[Tremor-Test Proband 4]{Gesamtabweichung und Abweichungen in den Raumachsen der Handflächenposition von Proband 4}
	\label{fig:Plot Tremor}
\end{figure}

\begin{table}[htb]
	\caption{Ergebnisse der maximalen Abweichungen der Handflächenposition}
	\label{table tremor}
	\begin{center}
		\begin{tabular}{|c|c|}
			\hline 
			$n$ & $d_{max}[mm]$ \\ 
			\hline 
			1 & 9,62 \\ 
			\hline 
			2 & 15,91 \\ 
			\hline 
			3 & 3,80 \\ 
			\hline 
			4 & 11,91 \\ 
			\hline 
			5 & 7,08 \\ 
			\hline 
			6 & 9,78 \\ 
			\hline 
			7 & 18,01 \\ 
			\hline 
			8 & 6,27 \\ 
			\hline 
			9 & 8,07 \\ 
			\hline 
			10 & 17,39 \\ 
			\hline 
		\end{tabular} 
	\end{center}
\end{table}

Abbildung \ref{fig:Plot Tremor} zeigt die Abweichungen bezüglich der Raumachsen und die resultierende Gesamtabweichung von Proband 4. Man erkennt, wie die Abweichungen anfangs mit der Zeit steigen. Gegen Ende der drei Sekunden sind keine Vergrößerungen der Abweichungen zu erkennen. Die Abweichung in y-Richtung sind bei Proband 4 stärker ausgeprägt, als bei anderen Probanden. In Hinblick auf Bewegungsrichtung der Hand konnte keine Präferenz unter den Probanden festgestellt werden.



\begin{table}[htb]
	\caption{Statistische Auswertung der Ergebnisse der maximalen Abweichungen der Handflächenposition}
	\label{table tremor2}
	\begin{center}
		\begin{tabular}{|c|c|c|c|c|c|}
			\hline 
			$r_{min}$ [mm] & $\sigma$ [mm] & RMSE [mm] & $r_{min,2D}$ [mm] & $\sigma_{2D}$ [mm] & RMSE$_{2D}$ [mm] \\
			\hline 
			10,78 & 4,90 & 11,74 & 8,81 & 4,00 & 9,59 \\ 
			\hline 
		\end{tabular} 
	\end{center}
\end{table}

In Tabelle \ref{table tremor} sind die maximalen Abweichungen aufgelistet. Die statistische Auswertung dieser Messergebnisse sind in Tabelle \ref{table tremor2} zu finden. Die geringste gemessene Abweichung beträgt 3,80 mm. Die größte gemessene Abweichung beträgt 18,01 mm. Die Spannweite dieser Messwerte beträgt 14,21 mm. Mit einer Standardabweichung von 4,90 mm schwanken die Werte unter den Probanden. Das arithmetische Mittel beträgt 10,78 mm. Dem Schwellwert $r_{min}$ wird dieser Wert zugewiesen. Entsprechend wird $r_{min,2D}$ durch Multiplikation von $r_{min}$ und $\frac{\sqrt{6}}{3}$ auf 8,80 mm bestimmt, da bezüglich der Bewegungsrichtung der Hand zwischen den drei Raumachsen kein Unterschied festgestellt wird. Der RMSE beträgt 11,74 mm. Die Standardabweichung $\sigma_{2D}$ und der RMSE$_2D$ fallen entsprechend geringer aus. Die Standardabweichung beträgt hier 4,00 mm während der RMSE einen Wert von 9,59 mm ergibt.

\begin{figure}[H]
	\centering
	\includegraphics[width=0.88\textwidth]{plot_auf_zu_alle.pdf}
	\caption[Erkennung der geöffneten/geschlossenen Hand]{Erkennung der geöffneten/geschlossenen Hand anhand der a) Anzahl ausgestreckter Finger, und \texttt{grabStrength}, b) Abstand zwischen Fingerspitzen und Handflächenmittelpunkt und c) Neigungswinkel der Fingerspitzen von Proband 1}
	\label{fig:plot_auf_zu_alle}
\end{figure}
\section{Handgeste zur Initiierung der Steuerung des Targets}



Um eine der vier Möglichkeiten, eine offene oder geschlossene Hand zu erkennen, auszuwählen, wurden alle entsprechenden Graphen, wie in Abbildung \ref{fig:plot_auf_zu_alle}, miteinander verglichen. Der Abstand zwischen Fingerspitzen und Handmittelpunkt schwankt bei jedem Finger unterschiedlich stark. Die Größten Schwankungen sind bei dem Mittelfinger zu erkennen. Vergleicht man die Testergebnisse hinsichtlich dieses Parameters, lässt sich kein absoluter Schwellwert bestimmen, der bei allen Probanden gleichzeitig und zuverlässig eine fehlerfreie Erkennung ermöglicht. 

Für die Neigungswinkel der Fingerspitzen lässt sich für alle Probanden ein Schwellwert ermitteln, der die Zustände der Hände eindeutig erkennt. Die Fingerspitzen weisen untereinander, mit Ausnahme des Daumens, stets sehr ähnliche Neigungswinkel auf. 

Die Anzahl der ausgestreckten Finger beschreibt nicht bei allen Personen den Handzustand korrekt. In zwei von zehn Durchläufen wurde ein ausgestreckter Finger erkannt, während die Hand geschlossen war. Die offene Hand wurde hingegen stets richtig erkannt. Setzt man die Schwelle, dass eine Hand als geschlossen gilt, auf weniger als zwei oder drei ausgestreckte Finger, so werden die Handzustände bei allen Probanden richtig erkannt.



Die \texttt{grabStrength} erkennt die Zustände der Hände aller Probanden fehlerfrei und wird gewählt. Man erkennt, dass der Wert der \texttt{grabStrength} sowohl beim Öffnen als auch beim Schließen der Hand erst nach einer halben Sekunde beginnt zu wechseln.



\section{Bestimmung der Öffnungszeit der Hand}

\begin{figure}[htb]
	\centering
	\includegraphics[width=0.8\textwidth]{plot_oeffnungszeit.pdf}
	\caption{Öffnungszeiten der Hand aller Probanden}
	\label{fig:Plot Öffnungszeit}
\end{figure}

\begin{table}[htb]
	\caption{Statistische Auswertung der Öffnungszeiten der Hand}
	\label{table oeffnungszeit}
	\begin{center}
		\begin{tabular}{|c|c|c|c|c|c|c|}
			\hline 
			Anzahl Messungen & $t_{min}$ [ms] & $t_{max}$ [ms] & ${t}_{open}$ [ms] & $\sigma$ [ms]] \\ 
			\hline 
			100 & 86 & 314 & 167,35 & 42,28 \\ 
			\hline 
		\end{tabular} 
	\end{center}
\end{table}



Von zehn Probanden wurden je zehn Handöffnungszeiten gemessen. Die Messwerte sind in einem Histogramm Abbildung \ref{fig:Plot Öffnungszeit} dargestellt. Anhand des Histogramms sieht man, dass die Werte normalverteilt sind. Wie in Tabelle \ref{table oeffnungszeit} zu erkennen, beträgt der Mittelwert dieser Normalverteilung 167,35 ms. Die zugehörige Standardabweichung beträgt 42,28 ms. Mit einer Handöffnungszeit von 86 ms wurde die schnellste Handöffnung gemessen. Der höchste gemessene Wert beträgt 314 ms.

\section{Virtuelle Positionierungsgenauigkeit}

\subsubsection{2D}
Zehn Probanden positionierten jeweils zwei Mal fünf Punkte mit drei unterschiedlichen Steuerungen. Bei der Drag\&Drop-Steuerung wurden pro Positionierung jeweils zwei Fehlerabweichungen, mit und ohne Fehlerkorrektur, gesammelt.
Die Fehlerabweichungen sind graphisch in Form mehrerer Histogramme, sortiert nach Steuerung und Versuch, in Abbildung \ref{fig:Plot Positionierung 2D} dargestellt.

\begin{figure}[h]
	\centering
	\includegraphics[width=0.75\textwidth]{plot_pos_2d.pdf}
	\caption[Testergebnisse der 2D Positionierung]{Testergebnisse der 2D Positionierung: Abweichungen zwischen Ist- und Sollposition für a) $S_{drag}(1,0ms)$, b) $S_{drag}(1,t_{open})$, c) $S_{joy,l}(1)$, d) $S_{joy,q}(1)$}
	\label{fig:Plot Positionierung 2D}
\end{figure}

\begin{table}[htb]
	\caption{Statistische Auswertung der Fehlerabweichungen der 2D-Positionierung durch verschiedene Steuerungen}
	\label{table pos 2d}
	\begin{center}
		\begin{tabular}{|c|c|c|c|c|c|c|c|}
			\hline 
			Steuerung & Versuch & $t$ & $e_{min}$ & $e$  & $\sigma$ & Median & RMSE \\ 
			&  & [s] & [mm] & [mm] & [mm] & [mm] & [mm] \\
			\hline 
			$S_{drag,2D}(1,0ms)=S_{d}$ & 1 & 5,40 & 3,61 & 11,63 & 6,26 & 9,93 & 13,18  \\ 
			\hline
			$S_{drag,2D}(1,0ms)=S_{d}$ & 2 & 5,72 & 0,31 & 9,34 & 5,19 & 8,60 & 10,65  \\ 
			\hline 
			$S_{drag,2D}(1,t_{open})=S_{dt}$ & 1 & 5,40 & 0,18 & 2,85 & 2,02 & 2,32 & 3,48 \\ 
			\hline 
			$S_{drag,2D}(1,t_{open})=S_{dt}$ & 2 & 5,72 & 0,24 & 2,30 & 1,93 & 1,63 & 2,98 \\ 
			\hline 
			$S_{joy,l,2D}(1)=S_{jl}$ & 1 & 6,73 & 0,13 & 4,68 & 3,79 & 3,81 & 6,00 \\ 
			\hline 
			$S_{joy,l,2D}(1)=S_{jl}$ & 2 & 5,37 & 0,17 & 4,24 & 3,54 & 3,47 & 5,50 \\ 
			\hline 
			$S_{joy,q,2D}(1)=S_{jq}$ & 1 & 6,03 & 0,32 & 2,84 & 2,12 & 2,52 & 3,53 \\ 
			\hline 
			$S_{joy,q,2D}(1)=S_{jq}$ & 2 & 6,53 & 0,41 & 2,78 & 2,17 & 2,42 & 3,51 \\ 
			\hline 
		\end{tabular} 
	\end{center}
\end{table}

Die Ausreißer in den Histogrammen kommen durch eine ungewollte Erkennung der offenen Hand und somit einer vorzeitigen Platzierung des Targets zustande. Die Anzahl der Ausreißer ist im Vergleich zur Anzahl der Messungen hoch. Besonders anfällig gegenüber Ausreißern ist der RMSE, da die Fehler quadriert werden. Um die Daten statistisch auszuwerten, wird ein Ausreißertest nach Grubbs durchgeführt. Das Signifikanzniveau von $\alpha=0.05$ wird für diesen Test gewählt. Dadurch werden für die statistische Auswertung bei $S_{d}$ jeweils drei Ausreißer vernachlässigt. Im ersten Versuch von $S_{dt}$ fallen fünf Werte, im zweiten Versuch zwei Werte weg. Für die restlichen Steuerungen werden, bis auf den zweiten Versuch von $S_{jq}$, bei der vier Ausreißer wegfallen, fünf Ausreißer vernachlässigt. Die durchschnittliche Zeit, die benötigt wurde, um das Target zu positionieren ist bei $S_{d}$ im ersten Versuch mit 5,40 Sekunden am geringsten. Die größte durchschnittliche Zeitspanne wurde bei $S_{jl}$ im ersten Versuch mit 6,73 Sekunden gemessen. Die geringste minimale Abweichung eines Durchlaufs zwischen Soll- und Istwert beträgt 0,13 mm und wurde mit $S_{jl}$ im ersten Versuch erzielt. Die kleinste durchschnittliche Abweichung beträgt 2,30 mm. Dieser Wert wurde mit  $S_{dt}$ im zweiten Versuch erreicht. Eine etwas höhere durchschnittliche Abweichung von 2,78 mm wurde mit $S_{jl}$ im zweiten Versuch erreicht. Die größte durchschnittliche Abweichung beträgt 11,63 mm und wurde mit $S_{d}$ im ersten Versuch erreicht. Der RMSE ist mit 13,18 mm und 10,65 mm bei der $S_{d}$ deutlich höher als bei den anderen Steuerungen. Mit $S_{dt}$ wurden mit 3,48 mm und 2,89 mm die geringsten Werte des RMSE erzielt. 

\subsubsection{3D}

\begin{table}[htb]
	\caption{Statistische Auswertung der Fehlerabweichungen der 3D-Positionierung durch verschiedene Steuerungen}
	\label{table pos 3d}
	\begin{center}
		\begin{tabular}{|c|c|c|c|c|c|c|c|}
			\hline 
			Steuerung & Versuch & $t$ & $e_{min}$ & $e$  & $\sigma$ & Median & RMSE \\ 
			&  & [s] & [mm] & [mm] & [mm] & [mm] & [mm] \\
			\hline 
			$S_{drag}(1,0ms)=S_{d}$ & 1 & 7,90 & 1,69 & 15,06 & 5,11 & 15,05 & 15,89 \\ 
			\hline
			$S_{drag}(1,0ms)=S_{d}$ & 2 & 6,70 & 2,92 & 14,25 & 8,40 & 12,03 & 16,49  \\ 
			\hline 
			$S_{drag}(1,t_{open})=S_{dt}$& 1 & 7,90 & 1,28 & 6,35 & 4,66 & 5,15 & 7,85  \\ 
			\hline 
			$S_{drag}(1,t_{open})=S_{dt}$ & 2 & 6,70 & 0,84 & 4,83 & 2,81 & 4,34 & 5,57  \\ 
			\hline 
			$S_{joy,l}(1)=S_{jl}$ & 1 & 10,86 & 2,51 & 16,20 & 15,04 & 10,76 & 22,00  \\ 
			\hline 
			$S_{joy,l}(1)=S_{jl}$ & 2 & 6,47 & 1,26 & 21,11 & 23,43 & 11,14 & 31,36  \\ 
			\hline 
			$S_{joy,q}(1)=S_{jq}$ & 1 & 9,33 & 0,39 & 8,27 & 7,01 & 5,38 & 10,79  \\ 
			\hline 
			$S_{joy,q}(1)=S_{j1}$ & 2 & 6,38 & 1,22 & 15,76 & 17,35 & 9,41 & 23,31  \\ 
			\hline 
		\end{tabular} 
	\end{center}
\end{table}

Die gemessenen Abweichungen zwischen Soll- und Istpositionen sind in mehreren Histogrammen in Abbildung \ref{fig:Plot Positionierung 3D} dargestellt. Auch hier sind, wie in der zweidimensionalen Positionierung, deutlich Ausreißer zu erkennen. Für die statistische Auswertung werden die Ausreißer mithilfe des Ausreißertest nach Grubbs erkannt und eliminiert. Das Signifikanzniveau von $\alpha=0.05$ wird wie bei der 2D-Positionierung gewählt. Bei $S_{dt}$ des ersten Versuches fallen sechs Ausreißer weg. Bei den restlichen Versuchen der Drag\&Drop-Steuerung fallen jeweils drei Ausreißer weg. Zwei Ausreißer des ersten Versuches und kein Ausreißer des zweiten Versuches werden bei $S_{jl}$ vernachlässigt. Bei $S_{jq}$ werden im ersten Versuch drei und im zweiten Versuch ein Ausreißer eliminiert. Die geringste durchschnittliche Zeit, die benötigt wurde, um das Target zu positionieren, wurde durch $S_{jq}$ im zweiten Versuch mit 6,38 Sekunden erzielt. Am längsten dauerte die Positionierung mit durchschnittlich 10,86 Sekunden bei der Joystick-Steuerung mit $S_{jl}$ im ersten Versuch. Die geringste minimale Abweichung wurde mit $S_{jq}$ erzielt. Die Abweichung beträgt 0,39 mm. Die geringsten durchschnittlichen Abweichungen wurden mit der $S_{dt}$ erreicht. Die Werte betragen im ersten Versuch 6,35 mm und im zweiten Versuch 4,83 mm. Der gleiche Test ohne Fehlerkorrektur ergab 15,06 mm im ersten und 14,25 mm im zweiten Versuch. Die größte durchschnittliche Abweichung mit 21,11 mm wurde mit $S_{jl}$ im zweiten Versuch erzielt. Die kleinsten RMSE Werte, mit 7,85 mm und 5,57 mm, wurden mit $S_{dt}$ erreicht. Die höchsten Werte hingegen wurden mit $S_{jl}$ erzielt. Die Werte betragen 22,00 mm im ersten und 31,36 mm im zweiten Versuch.

\begin{figure}[htb]
	\centering
	\includegraphics[width=0.75\textwidth]{plot_pos_3d.pdf}
	\caption[Testergebnisse der 3D Positionierung]{Testergebnisse der 3D Positionierung: Abweichungen zwischen Ist- und Sollposition für a) $S_{drag}(1,0ms)$, b) $S_{drag}(1,t_{open})$, c) $S_{joy,l}(1)$, d) $S_{joy,q}(1)$}
	\label{fig:Plot Positionierung 3D}
\end{figure}


\section{Gestensteuerung des Targets entlang von Pfaden}
Der zeitliche Anteil auf dem Pfad wird, wie in Formel \ref{eq:pfad} beschrieben, bestimmt.
\begin{equation}\label{eq:pfad}
t_{Pfad}=\frac{t_{on}}{t_{on}+t_{off}}
\end{equation}

\subsection{Virtueller Pfad}

Die Ergebnisse der einzelnen Probanden des virtuellen Pfades durch die Drag\&Drop-Steuerung mit Korrektur sind in Tabelle \ref{table pfad v d} dargestellt. Die entsprechenden Ergebnisse der unterschiedlichen Joystick-Steuerungen sind in den Tabellen \ref{table pfad v k0} und \ref{table pfad v k1} zu finden. Die Verläufe der Positionen aller getesteten Steuerungen sind in Abbildung \ref{fig:plot_pfad_pos} dargestellt.

\begin{figure}[H]
	\centering
	\includegraphics[width=0.9\textwidth]{plot_pfad_pos.pdf}
	\caption[Positionen entlang des virtuellen Pfades aller Probanden]{Positionen entlang des virtuellen Pfades aller Probanden bei unterschiedlichen Steuerungen a) $S_{drag}(0.1,t_{open})$, b) $S_{joy,l}(0.1)$, $S_{joy,q}(0.1)$}
	\label{fig:plot_pfad_pos}
\end{figure}

\subsubsection{Drag\&Drop-Steuerung}
Die Spannweite der Positionsabweichungen stellt mit 3,98 mm den kleinsten und mit 12,07 mm den höchsten gemessenen Wert dar. Der RMSE nimmt mit 0,75 mm den kleinsten Wert ein. Der höchste gemessene Wert des RMSE beträgt 3,52 mm. Der Proband, der während des Tests den geringsten zeitlichen Anteil auf dem Pfad verbracht hat, erreichte einen prozentualen Anteil von 19,54\%. Ein anderer Proband hingegen erreichte mit 99,04\% den höchsten zeitlichen Anteil mit dieser Steuerung. Die Zeit, die für die Absolvierung des Pfades benötigt wurde, reicht von 28,46 Sekunden bis 64,05 Sekunden. 

\begin{table}[htb]
	\caption{Ergebnisse des virtuellen Pfades durch $S_{drag,2D}(0.1,t_{open})$}
	\label{table pfad v d}
	\begin{center}
		\begin{tabular}{|c|c|c|c|c|c|}
			\hline 
			Proband & Spannweite [mm] & $\sigma$ [mm] & RMSE [mm] & $t_{Pfad}$ [\%] & $t$ [s] \\ 
			\hline 	
			1 & 4,49 & 0,69 & 0,92 & 98,91 & 43,8  \\ 
			\hline
			2 & 6,47 & 1,25 & 3,17 & 19,54 & 58,7  \\ 
			\hline
			3 & 4,79 & 0,55 & 0,75 & 99,04 & 36,5  \\ 
			\hline 
			4 & 12,07 & 2,71 & 3,34 & 36,27 & 28,46  \\ 
			\hline 
			5 & 10,20 & 2,68 & 3,52 & 68,62 & 36,54  \\ 
			\hline 
			6 & 6,88 & 1,09 & 1,14 & 97,02 & 64,05  \\ 
			\hline 
			7 & 3,98 & 0,82 & 1,84 & 82,38 & 62,38  \\ 
			\hline 
			8 & 8,98 & 1,76 & 2,63 & 79,13 & 50,10  \\ 
			\hline 
			9 & 10,16 & 1,37 & 1,37 & 91,01 & 32,72  \\ 
			\hline 
			10 & 7,75 & 0,84 & 1,04 & 96,85 & 33,12  \\ 
			\hline 
		\end{tabular} 
	\end{center}
\end{table}

\subsubsection{Joystick-Steuerung mit linearem Geschwindigkeitsverlauf}
Die Spannweite der Fehlerabweichungen reicht bei dieser Steuerung von 0,96 mm bis 5,43 mm. Der RMSE hat mit 0,29 mm den niedrigsten und mit 1,22 mm den höchsten gemessenen Wert. Bei acht von zehn Probanden beträgt der zeitliche Anteil, der auf dem Pfad verbracht wurde 100\%. Die anderen zwei Probanden erreichten Werte von 77,65\% und 93,32\%. Die Zeit, die benötigt wurde, um den Pfad zu absolvieren, reicht von 15,31 Sekunden bis 25,08 Sekunden.

\begin{table}[htb]
	\caption{Ergebnisse des virtuellen Pfades durch $S_{joy,l,2D}(0.1)$}
	\label{table pfad v k0}
	\begin{center}
		\begin{tabular}{|c|c|c|c|c|c|}
			\hline 
			Proband & Spannweite [mm] & $\sigma$ [mm] & RMSE [mm] & $t_{Pfad}$ [\%] & Zeit [s] \\ 
			\hline 	
			1 & 2,14 & 0,50 & 0,66 & 100 & 23,88  \\ 
			\hline
			2 & 2,39 & 0,52 & 0,80 & 100 & 20,07  \\ 
			\hline
			3 & 0,96 & 0,25 & 0,29 & 100 & 24,57  \\ 
			\hline 
			4 & 2,84 & 0,98 & 1,15 & 100 & 19,79  \\ 
			\hline 
			5 & 3,03 & 0,90 & 1,17 & 93,32 & 20,49  \\ 
			\hline 
			6 & 3,92 & 1,22 & 1,22 & 100 & 20,96  \\ 
			\hline 
			7 & 5,43 & 1,65 & 1,98 & 77,65 & 25,08  \\ 
			\hline 
			8 & 1,69 & 0,44 & 0,45 & 100 & 19,25  \\ 
			\hline 
			9 & 1,17 & 0,26 & 0,29 & 100 & 16,04  \\ 
			\hline 
			10 & 1,26 & 0,31 & 0,40 & 100 & 15,31  \\ 
			\hline 
		\end{tabular} 
	\end{center}
\end{table}

\subsubsection{Joystick-Steuerung mit quadratischem Geschwindigkeitsverlauf}
Die Spannweite der Abweichungen reicht bei dieser Steuerung von 0,72 mm bis 6,71 mm. Der kleinste gemessene RMSE beträgt 0,22 mm. Der höchste RMSE hingegen beträgt 1,83 mm. Acht von zehn Probanden erreichten einen hundertprozentigen zeitlichen Anteil auf dem Pfad. Die anderen zwei Probanden erzielten Werte von 81,89\% und 90,77\%. Die Zeit, die benötigt wurde, um den Pfad zu absolvieren beträgt bei dem schnellsten Probanden 5,14 Sekunden. Dieser Wert ist die kürzeste Zeitspanne hinsichtlich aller getesteten Steuerungen. Die längste Zeit, die ein Proband mit dieser Steuerung benötigte, um den Pfad zu absolvieren, beträgt 26,97 Sekunden.

\begin{table}[htb]
	\caption{Ergebnisse des virtuellen Pfades durch $S_{joy,q,2D}(0.1)$}
	\label{table pfad v k1}
	\begin{center}
		\begin{tabular}{|c|c|c|c|c|c|}
			\hline 
			Proband & Spannweite [mm] & $\sigma$ [mm] & RMSE [mm] & $t_{Pfad}$ [\%] & $t$ [s] \\ 
			\hline 	
			1 & 0,72 & 0,21 & 0,25 & 100 & 10,61  \\ 
			\hline
			2 & 4,99 & 1,79 & 1,80 & 90,77 & 26,97  \\ 
			\hline
			3 & 0,82 & 0,18 & 0,22 & 100 & 10,02  \\ 
			\hline 
			4 & 2,10 & 0,66 & 0,85 & 100 & 11,87  \\ 
			\hline 
			5 & 6,71 & 1,69 & 1,83 & 81,89 & 15,40  \\ 
			\hline 
			6 & 3,29 & 1,10 & 1,10 & 100 & 8,33  \\ 
			\hline 
			7 & 3,17 & 0,73 & 0,91 & 100 & 9,39  \\ 
			\hline 
			8 & 2,07 & 0,38 & 0,45 & 100 & 9,40  \\ 
			\hline 
			9 & 2,23 & 0,65 & 0,76 & 100 & 5,14  \\ 
			\hline 
			10 & 1,53 & 0,36 & 0,45 & 100 & 13,04  \\ 
			\hline 
		\end{tabular} 
	\end{center}
\end{table}


\subsection{Reale Pfade}
In Tabelle \ref{table pfad rob1} sind die Ergebnisse der UR3-Steuerung entlang des ersten Pfades aufgelistet. In Tabelle \ref{table pfad rob2} sind die entsprechenden Ergebnisse des zweiten Pfades zu finden. Die statistische Auswertung ist in Tabelle 

\subsubsection{Pfad 1}

\begin{table}[htb]
	\caption{Ergebnisse der UR3-Steuerung entlang des ersten Pfades durch $S_{drag,2D}(0.1,t_{open})$}
	\label{table pfad rob1}
	\begin{center}
		\begin{tabular}{ |c|c|c|c| }
			\hline
			Proband & Versuch & $t_{Pfad}$ [\%] & $t$ [s] \\ 
			\hline
			\multirow{2}{*}{1} 
			& 1 & 80,27 & 38,39 \\
			& 2 & 90,89 & 30,19 \\
			\hline
			
			\multirow{2}{*}{2} 
			& 1 & 69,92 & 53,52 \\
			& 2 & 80,67 & 26,58 \\
			\hline	
			
			\multirow{2}{*}{3} 
			& 1 & 55,58 & 91,30 \\
			& 2 & 63,07 & 64,10 \\
			\hline
			
			\multirow{2}{*}{4} 
			& 1 & 87,00 & 156,61 \\
			& 2 & 81,48 & 157,66 \\
			\hline
			
			\multirow{2}{*}{5} 
			& 1 & 96,47 & 40,38 \\
			& 2 & 99,54 & 37,20 \\
			\hline
			
			\multirow{2}{*}{6} 
			& 1 & 69,06 & 38,46 \\
			& 2 & 98,05 & 45,19 \\
			\hline
			
			\multirow{2}{*}{7} 
			& 1 & 61,18 & 82,84 \\
			& 2 & 97,49 & 52,30 \\
			\hline
			
			\multirow{2}{*}{8} 
			& 1 & 82,85 & 39,31 \\
			& 2 & 91,71 & 61,63 \\
			\hline
			
			\multirow{2}{*}{9} 
			& 1 & 80,82 & 48,52 \\
			& 2 & 83,80 & 63,96 \\
			\hline
			
			\multirow{2}{*}{10} 
			& 1 & 80,21 & 47,94 \\
			& 2 & 89,94 & 53,32 \\
			\hline
		\end{tabular}
	\end{center}
\end{table}

\begin{figure}[H]
	\centering
	\includegraphics[width=0.85\textwidth]{plot_pfad_rob_1.pdf}
	\caption{Roboterpositionen entlang des ersten Pfades aller Probanden}
	\label{fig:plot_pfad_rob_1}
\end{figure}

Die Zeiten, welche die Probanden benötigten den Pfad zu beenden, schwanken sowohl im ersten als auch im zweiten Versuch stark. Die Zeiten betragen im ersten Versuch zwischen 38,39 Sekunden und 156,61 Sekunden. Im zweiten Versuch sind es zwischen 26,58 Sekunden und 157,66 Sekunden. Die maximalen Zeiten sind dabei von demselben Probanden erzielt worden. Keiner der Probanden erreichte einen zeitlichen Anteil von 100\% auf dem Pfad. Der höchste Anteil wurde mit 99,54\% im zweiten Versuch erreicht. Mit 55,58\% wurde der geringste zeitliche Anteil im ersten Versuch auf dem Pfad gemessen.

\subsubsection{Pfad 2}

Um den Roboter entlang des zweiten Pfades zu steuern benötigten die Probanden im ersten Versuch zwischen 66,61 Sekunden und 344,19 Sekunden. Im zweiten Versuch sind es zwischen 65,34 und 447,75 Sekunden. Auch bei diesem Test sind die maximalen Zeiten von demselben Probanden erzielt worden. Der höchsten zeitliche Anteil auf dem Pfad wurde mit 99,3\% im zweiten Versuch erreicht, während der geringste Anteil im ersten Versuch mit 63,07\% erzielt wurde. In Abbildung \ref{fig:plot_pfad_rob_2} ist sind die eckigen und runden Bereiche des Pfades sehr gut zu erkennen.

\begin{figure}[H]
	\centering
	\includegraphics[width=1.0\textwidth]{plot_pfad_rob_2.pdf}
	\caption{Roboterpositionen entlang des zweiten Pfades aller Probanden}
	\label{fig:plot_pfad_rob_2}
\end{figure}

\begin{table}[htb]
	\caption{Ergebnisse der UR3-Steuerung entlang des zweiten Pfades durch $S_{drag,2D}(0.1,t_{open})$}
	\label{table pfad rob2}
	\begin{center}
		\begin{tabular}{ |c|c|c|c| }
			\hline
			Proband & Versuch & $t_{Pfad}[\%]$ & $t$ [s] \\ 
			\hline
			\multirow{2}{*}{1} 
			& 1 & 96,14 & 100,18 \\
			& 2 & 96,50 & 76,08 \\
			\hline
			
			\multirow{2}{*}{2} 
			& 1 & 92,73 & 66,61 \\
			& 2 & 89,98 & 65,34 \\
			\hline	
			
			\multirow{2}{*}{3} 
			& 1 & 60,74 & 169,92 \\
			& 2 & 76,92 & 172,52 \\
			\hline
			
			\multirow{2}{*}{4} 
			& 1 & 87,79 & 344,19 \\
			& 2 & 83,55 & 447,75 \\
			\hline
			
			\multirow{2}{*}{5} 
			& 1 & 97,61 & 94,87 \\
			& 2 & 99,31 & 99,69 \\
			\hline
			
			\multirow{2}{*}{6} 
			& 1 & 89,17 & 74,02 \\
			& 2 & 93,24 & 90,79 \\
			\hline
			
			\multirow{2}{*}{7} 
			& 1 & 94,13 & 144,15 \\
			& 2 & 99,08 & 130,82 \\
			\hline
			
			\multirow{2}{*}{8} 
			& 1 & 86,35 & 99,52 \\
			& 2 & 89,39 & 93,01 \\
			\hline
			
			\multirow{2}{*}{9} 
			& 1 & 78,03 & 143,32 \\
			& 2 & 84,55 & 142,27 \\
			\hline
			
			\multirow{2}{*}{10} 
			& 1 & 89,22 & 128,76 \\
			& 2 & 98,48 & 122,13 \\
			\hline
		\end{tabular}
	\end{center}
\end{table}

\subsection{Gemittelte Ergebnisse der Pfad-Versuche}
\subsubsection{Virtueller Pfad}
Die gemittelten Ergebnisse des virtuellen Pfades sind in Tabelle \ref{table pfad all} dargestellt. Die durchschnittliche Spannweite der Abweichungen ist mit 7,57 mm bei $S_{d}$ am höchsten. Mit 2,48 mm und 2,76 mm sind die Werte der Spannweite durch $S_{jl}$ und $S_{jq}$ deutlich geringer. Der RMSE bei $S_{d}$ beträgt 1,97 mm und stellt damit den höchsten Wert dar. Durch die Joystick-Steuerungen wurden ähnliche Werte von 0,84 mm und 0,86 mm erzielt. Auch bei den zeitlichen Anteilen auf dem Pfad sind die Werte der Joystick-Steuerungen mit 97,10\% und 97,27\% sehr ähnlich. Durch$S_{d}$ hingegen wurde ein zeitlicher Anteil von 76,88\% auf dem Pfad erreicht. Durchschnittlich benötigten die Probanden mit dieser Steuerung 41,32 Sekunden. Deutlich weniger Zeit wurde mit $S_{jl}(1)$ benötigt. Mit dieser Steuerung nahmen die Probanden durchschnittlich 19,01 Sekunden in Anspruch, um den Pfad zu absolvieren. Die kürzeste Zeitspanne wurde mit $S_{jq}$ benötigt. Die Probanden durchliefen den Pfad mit einer durchschnittlichen Zeit von 10,71 Sekunden.


\begin{table}[htb]
	\caption{Gemittelte Ergebnisse des virtuellen Pfades der getesteten Steuerungen}
	\label{table pfad all}
	\begin{center}
		\begin{tabular}{ |c|c|c|c|c|c| }
			\hline
			Steuerung & Spannweite [mm] & $\sigma$ [mm] & RMSE & $t_{Pfad}$  & $t$ [s] \\
			\hline
			$S_{drag,2D(0.1,t_{open})}=S_{d}$ & 7,57 & 1,38 & 1,97 & 76,88 & 41,32 \\
			\hline
			$S_{joy,l,2D(0.1)}=S_{jl}$ & 2,48 & 0,70 & 0,84 & 97,10  & 19,01\\
			\hline
			$S_{joy,q,2D(0.1)}=S_{jq}$ & 2,76 & 0,77 & 0,86  & 97,27 & 10,71\\
			\hline
		\end{tabular}
	\end{center}
\end{table}

\subsubsection{Reale Pfade}
Die gemittelten Ergebnisse der realen Pfade sind in Tabelle \ref{table pfad all_rob} dargestellt. Die zeitlichen Anteile auf dem Pfad liegen zwischen 76,33 und 91,10\%. Der zweite Versuch ist in dieser Hinsicht auf beiden Pfaden deutlich besser ausgefallen. 
\begin{table}[h]
	\caption{Gemittelte Ergebnisse der realen Pfade durch $S_{drag,2D(0.1,t_{open})}$}
	\label{table pfad all_rob}
	\begin{center}
		\begin{tabular}{ |c|c|c|c| }
			\hline
			Versuch & Pfad & $t_{Pfad}$ [\%]  & $t$ [s] \\
			\hline
			1 & 1 & 76,33 & 63,72\\
			\hline
			1 & 2 & 87,19 & 136,55\\
			\hline
			2 & 1 & 87,66 & 59,21\\
			\hline
			2 & 2& 91,10 & 144,04\\
			\hline
		\end{tabular}
	\end{center}
\end{table}

\subsection{Kalibrierung}

\begin{figure}[H]
	\centering
	\includegraphics[trim= 0cm 12cm 0cm 8cm, clip,width=15cm]{calibrationPixelMM.jpg}
	\caption{Abweichung von Ist- und Sollwert zur Bestimmung von $s_{x}$}
	\label{fig:calibrationPixelMM}
\end{figure}

Wie in \ref{fig:calibrationPixelMM} zu sehen, wurde die Skalierung mit Hilfe eines Lineals und der Mikroskopkamera ermittelt und unterscheiden sich dabei in x-Richtung und y-Richtung. Folgende Werte wurden für die Skalierung in x-Richtung  und in y-Richtung ermittelt:

\begin{itemize}
	\item
	$s_{x}=\frac{6378 Pixel}{(270-0,544)mm}=23,6699\frac{Pixel}{mm}$
	\item
	$s_{y}=\frac{4488 Pixel}{(190-0,858)mm}=23,7282\frac{Pixel}{mm}$.
\end{itemize}	

\subsection{Parallelität zwischen Roboterebene und Pfad}

\begin{figure}[h]
	\centering
	\includegraphics[width=0.9\textwidth,angle=0]{plot_ebenen_rotiert.pdf}
	\caption{Fehler in Abhängigkeit des Winkels zwischen den Ebenen}
	\label{fig:Plot Fehler Ebenen rotiert}
\end{figure}

Der Versatz in y-Richtung zwischen den Mittelpunkten der Mikroskopkamerabilder und den Kalibrierpunkten beträgt ca. zwischen 0,5 mm und 0,7 mm.
Wie in Abbildung \ref{fig:Plot Fehler Ebenen rotiert} zu sehen, sind die Abweichungen $e_{1}$ und $e_{2}$ bei kleinen Winkeln nahezu identisch.
Bei einer Abweichung von 0.622 mm wird auf einen Winkel von 0,86$^{\circ}$ geschlossen. Da die Kalibrierung zu diesem Zeitpunkt bereits nach \eqref{eq:frobenius} einen verhältnismäßig kleinen Fehler von 0,71 mm aufweist, ist die Transformationsmatrix für diese Zwecke genau genug. Man beachte hierbei, dass sich die Pfade mittig auf dem Blatt befinden, während die Kalibrierpunkte außen platziert sind. Die Abweichungen werden somit zur Blattmitte geringer. Bei dem Anfahren verschiedener Punkte auf den Pfaden ließen sich kaum noch Abweichungen feststellen.



\section{Latenzen}
\subsection{Mikroskopkamera}
Die Verschlusszeit der Fotokamera betrug bei den Messungen $\frac{1}{2000}$ Sekunden.

Mithilfe des VLC-Players wird untersucht, welche Video-Modi der Mikroskopkamera unterstützt werden. 
\begin{center}
	\begin{tabular}{|c|c|}
		\hline 
		Auflösung & Framerate [fps] \\ 
		\hline 
		640x480 & 30 \\ 
		\hline 
		800x600 & 25 \\ 
		\hline 
		1280x960 & 10 \\ 
		\hline 
		1600x1200 & 5 \\ 
		\hline 
		2048x1536 & 5 \\ 
		\hline 
	\end{tabular} 
\end{center}
Zur Ermittlung der durchschnittlichen Latenzzeit zwischen Ereignis- und Darstellungszeitpunkt wurden 35 Fotos geschossen. 21 dieser Fotos konnten nicht ausgewertet werden, da die Zehntelsekunden der Zeitanzeige nicht lesbar waren. Die Hundertstelsekunden der Stoppuhr waren im Gegensatz dazu auf jedem Foto zu erkennen. Die Bildwechelfrequenz der Mikroskopkamera wurde auf ihr Maximum von 30 fps eingestellt. Die gemessenen Zeitdifferenzen schwanken stark. Die Werte reichen von 23 bis 85 ms. Die Auswertung der Messungen sind in Tabelle \ref{table latenz cam} zu finden.
%a=[23 28 32 38 40 40 48 78 70 38 51 61 50 85] Messwerte

\begin{table}[htb]
	\caption{Auswertung der Messungen zur Bestimmung der Latenz der Mikroskopkamera}
	\label{table latenz cam}
	\begin{center}
		\begin{tabular}{|c|c|c|c|c|}
			\hline 
			 $n_{ges}$ & $t_{min}$ [ms] & $t_{max}$ [ms] & $\bar{t}$ [ms] & $\sigma_{t}$ [ms] \\ 
			\hline 
			14 & 23 & 85 & 48,71 & 18,68 \\ 
			\hline 
		\end{tabular} 
	\end{center}
\end{table}
\begin{figure}[H]
	\centering
	\includegraphics[width=0.9\textwidth]{plot_rob_latenz2.pdf}
	\caption{Ermittlung der zeitlichen Differenz zwischen Hand- und Roboterbewegung}
	\label{fig:plot_rob_latenz}
\end{figure}

\subsection{UR3-Roboter}



Während die Probanden den Roboter entlang der Pfade steuerten, wurden die Pixelpositionen und Handpositionen der Leap Motion API zu jedem Zeitpunkt gespeichert. Anhand dieser Daten kann die Latenz des Roboters bestimmt werden, da die Pixelpositionen durch die Roboterposition bestimmt werden. Wie in \ref{fig:plot_rob_latenz} dargestellt, wurden die Zeitdifferenzen ermittelt . Es wurden Differenzen zwischen 20 und 40 ms abgelesen. Anhand von fünf Messungen wird eine durchschnittliche Latenz des Roboters von 28 ms bestimmt.
\documentclass[
a4paper, 
%a5paper,
%10pt,
%11pt,
12pt,
%twoside, % single sided printout
%oneside, % duplex printout (default)
%% binding correction is used to compensate for the paper lost during binding
%% of the document
%BCOR=0.7cm, % binding correction
%nobcorignoretitle, % do not ignore BCOR for title page
%% the following two options only concern the graphics included by the document
%% class
%grayscaletitle, % keep the title in grayscale
grayscalebody, % keep the rest of the document in grayscale
abstract=on,
%% expert options: your mileage may vary
%baseclass=scrreprt % special option to use a different document baseclass
twoside, BCOR10mm, 12pt, DIV13,headinclude, footexclude, final, abstracton, openright
]{ibireprt}

\usepackage{array}
\usepackage[toc]{appendix}
%\usepackage{gensymb}
\usepackage[utf8]{inputenc}
%\usepackage[latin1]{inputenc}
\usepackage[T1]{fontenc}
%\usepackage{ngerman}
%\usepackage[ngerman]{babel} %english,
\usepackage[english]{babel} %english,
%\usepackage{fancyhdr}
%%\pagestyle{fancy}
%%\lhead{\leftmark}
%\fancyhead[OR]{\thepage}% ungerade Seiten, rechts \thepage
%\fancyhead[OL]{\leftmark}% ungerade Seiten, links
%\fancyhead[ER]{\leftmark}% gerade Seiten, rechts
%\fancyhead[EL]{\thepage}% gerade Seiten, links \thepage
%\cfoot{}
%\pagestyle{fancy}

\renewcommand{\chaptermark}[1]{\markboth{#1}{}}
\renewcommand{\sectionmark}[1]{\markright{#1}{}}

\setlength{\parindent}{0em}
%\setlength{\parskip}{0mm}
\setcounter{tocdepth}{1}
\setcounter{secnumdepth}{2}
%\linespread{1.15}

\usepackage{quotchap}
\usepackage{nccmath}

\usepackage{microtype}
\usepackage{qtxmath,tgtermes}
\usepackage[scaled=.90]{helvet}
\usepackage{courier}
\usepackage{graphicx}
\usepackage{amsmath}
%\usepackage[linesnumbered,ruled]{algorithm2e}
%\usepackage{amsfonts}
\usepackage{amssymb}
\usepackage{tabularx}
%\usepackage[bookmarks,plainpages=false]{hyperref} %colorlinks  ,urlbordercolor={111},linkbordercolor={111},citebordercolor={111}
%\usepackage[Algorithmus]{algorithm}

\usepackage{algorithm}
\usepackage{algorithmic}
%\usepackage[noend]{algpseudocode}
%\usepackage{tkmath}
\usepackage{exscale}
\usepackage{empheq}
\usepackage{color}
\usepackage{framed}
\usepackage{rotating}
\usepackage{longtable}
\usepackage[hang,small,bf]{caption}
\usepackage{booktabs}
\usepackage{colortbl}
\usepackage{subcaption}
\usepackage[babel,german=quotes]{csquotes}
\usepackage{ntheorem}
\usepackage{blindtext}
\usepackage{url}


%\mathindent1.5cm
\def\fleq{\@fleqntrue \let\mathindent\@mathmargin \@mathmargin=\leftmargini}
\def\cneq{\@fleqnfalse}
%\setcapindent{0em}
\newenvironment{fshaded}{%
	\def\FrameCommand{\fcolorbox{framecolor}{shadecolor}}%
	\MakeFramed {\FrameRestore}}%
{\endMakeFramed}

\theoremseparator{:}

\newtheorem{theorem}{Theorem}[chapter]
\newtheorem{lemma}{Lemma}[chapter]
\newtheorem{remark}[theorem]{Bemerkung}
\newtheorem{definition}[theorem]{Definition}
\newtheorem{example}{Beispiel}
%\newtheorem{proof}[theorem]{Beweis}
\newtheorem{corollary}[theorem]{Corollary}

\newenvironment{Theorem}{\goodbreak \definecolor{shadecolor}{rgb}{0.95,0.95,0.95}%
	\definecolor{framecolor}{rgb}{0,0,0}%
	\begin{fshaded}\begin{theorem}\sl}{\end{theorem} \end{fshaded}}
\newenvironment{Lemma}{\goodbreak \definecolor{shadecolor}{rgb}{0.95,0.95,0.95}%
	\definecolor{framecolor}{rgb}{0,0,0}%
	\begin{fshaded} \begin{lemma}\sl}{\end{lemma} \end{fshaded}}
\newenvironment{Remark}{\goodbreak \begin{remark}\rm}{\hfill  $\square$\end{remark}}
\newenvironment{Example}{\goodbreak \begin{example}\rm}{\hfill $\square$ \end{example}}
%\newenvironment{Proof}{\goodbreak \begin{proof}\rm}{\hfill $\blacksquare$ \end{proof}}
\newenvironment{Definition}{\goodbreak \definecolor{shadecolor}{rgb}{0.95,0.95,0.95}%
	\definecolor{framecolor}{rgb}{0,0,0}%
	\begin{fshaded} \begin{definition}\rm}{\hfill  \end{definition} \end{fshaded} }
\newenvironment{Corollary}{\goodbreak \begin{corollary}\rm}{\end{corollary}}

\newenvironment{Proof}[1][Beweis:]{\begin{trivlist}
		\item[\hskip \labelsep {\bfseries #1}]}{\hfill $\blacksquare$\end{trivlist}}

\numberwithin{equation}{chapter}
\numberwithin{table}{chapter}
\numberwithin{figure}{chapter}
\numberwithin{algorithm}{chapter}
\numberwithin{example}{chapter}
\numberwithin{example}{chapter}

\def\i{\mbox{\small{\rm i}}}
\def\ti{\mbox{\scriptsize{\rm i}}}
\newcommand{\e}[1]{{\rm e}^{ #1}}
\renewcommand{\mod}{\;{\rm mod}\;}

\newcommand{\zb}[1]{\mbox{\boldmath{${#1}$}}}
\newcommand{\zbs}[1]{\mbox{\boldmath\scriptsize{${#1}$}}}
\newcommand{\zbss}[1]{\mbox{\boldmath\tiny{${#1}$}}}

\newcommand{\adj}{{\ensuremath{\mathsf{H}}}}
\newcommand{\trans}{{\ensuremath{\mathsf{T}}}}


% Keine "Schusterjungen"
\clubpenalty = 10000
% Keine "Hurenkinder"
\widowpenalty = 10000 \displaywidowpenalty = 10000 %\displaywidowpenalty = 10000






% Information for the Titlepage
\author{Fabian Hielscher}
\title{Image registration dings mit machine learning}
\date{\today}
\subject{Master's Thesis}
\professor{Prof.~Dr.-Ing.~Tobias Knopp}
\advisor{?}


\begin{document}
	%\frontmatter
	\maketitle
	%\mainmatter
	
	\newpage
	${}^{}$
	\vfill
	\noindent
	Ich versichere an Eides statt, die vorliegende Arbeit selbstständig und nur unter Benutzung der angegebenen Quellen und Hilfsmittel angefertigt zu haben.\\
	\vspace{1.5cm}
	
	\noindent
	Hamburg, den ??.??.2010
	\thispagestyle{empty}
	\newpage
	\newpage
	
	\setlength{\parskip}{1.5mm }
	
	%\newpage
	
	%\maketitle
	
	
	\setcounter{tocdepth}{2}
	\tableofcontents
	\begin{abstract}
	
	\end{abstract}
	
	\chapter{Introduction}
%	\begin{table}[h!]
%		\begin{center}
%			\begin{tabular}{l|l|l|l} % <-- Alignments: 1st column left, 2nd middle and 3rd right, with vertical lines in between
%				\textbf{Experiment} & \textbf{Preprocessing} & \textbf{Undersampling} & \textbf{Hyperparameter} \\
%				\hline
%				$E_1$ - Input data preprocessing & Evaluation & None & Default \\
%				$E_2$ - Undersampling masks & Result from $E_1$& Evaluation & Default\\
%				$E_3$ - Hyperparameter tuning & Result from $E_1$ & Result from $E_2$ & Evaluation \\
%				$E_4$ - CS comparison & Result from $E_1$ & Result from $E_2$ & Result from $E_3$ \\
%			\end{tabular}
%			\caption{Overview of performed experiments.}
%			\label{tab:Overview of performed experiments.}
%		\end{center}
%	\end{table}
	\chapter{Theory}
	\section{Parametric image registration}
		\subsection{Landmark-based registration}
		\subsection{Principal axes-based registration}
		\subsection{Optimal linear registration}
			\subsubsection{Intensity-based registration}
			\subsubsection{Correlation-based registration}
			\subsubsection{Mutual information-based registration}
	\section{Non-parametric image registration}
		\subsection{Elastic registration}
		\subsection{Fluid registration}
		\subsection{Diffusion registration}	
		\subsection{Curvature registration}	
	\section{Reinforcement learning}
	In Reinforcement Learning, an agent is supposed to learn a specific behavior by trial-anderror. The agent interacts with its environment to collect experiences. Figure 2.6 illustrates the basic concept behind Reinforcement Learning. At each time step t = 1, 2, 3, ...
	the agent is in a certain state $s \in \mathcal{S}$ and takes one of the possible available actions
	$a \in \mathcal{A}(s)$. The action changes the environment and the agent ends up in a new state
	st+1. Furthermore, it receives a reward Rt+1 from the environment that serves as a feedback about how good it was to take action at
	in state st
	
	\subsection{Markov Decision Process}
	The Markov Assumption assumes an independence of past and future states, meaning
	that the state and the behavior of the environment at time step t are not 
	influenced by the past agent-environment interactions
	$a_1, ..., a_{t-1}$
	If the RL-task can fulfill the Markov Assumption, it can be formulated as five-tuple Markov Decision Process $(\mathcal{S},\mathcal{A}, \mathcal{P}_{s,s'}^a, \mathcal{R}_{s,s'}^a, \gamma)$
	
	\begin{itemize}
		\item $\mathcal{S}$: set of states
		\item $\mathcal{A}$: set of actions. $\mathcal{A}(s)$ is the best action in state s.
		\item $\mathcal{P}_{s,s'}^a$: Transition probabilities. It is the probability of the	transition from s to s' when taking action a in state s at time step t.
		\item $\mathcal{R}_{s,s'}^a$: Reward probabilities. It defines the immediate reward the agent receives after the transition from s to s'
		\item  $\gamma$: Discount factor.$ \gamma \in [0, 1]$ for computing the discounted expected return
	\end{itemize}
	
	\chapter{Experimental Setup}
			
	\chapter{Discussion}
		
	
	\chapter{Conclusion}
	
\end{document}
